\chapter{Introduction}
%\chapter{Einleitung}

% \begin{itemize}
% 	\item general motivation for your work, context and goals.
% 	\item context: make sure to link where your work fits in
% 	\item problem: gap in knowledge, too expensive, too slow, a deficiency, superseded technology
% 	\item strategy: the way you will address the problem
% 	\item recommended length: 1-2 pages.
% \end{itemize}

\section*{Motivation/Requirements}
% \begin{itemize}
% 	\item Reliability
% 	\item .. and why 100\% Reliability is not important (except pyrotechnics)
% 	\item Lower latency
% 	\item Synchronisation
% 	\item higher update frequency
% 	\item Range
% \end{itemize}

Wireless networks are becoming increasingly popular, also in the field of lighting technology,
because they can be more flexible and also more cost-effective.
However, they also have limitations compared to wired solutions.
In order to improve the properties of radio networks, a logic link layer approach is chosen in this thesis.
to distribute the control signals to the individual stations,
This promises better performance due to reduced overhead.

The properties of the wired DMX-512a solution are taken as a baseline.
The packets are to be distributed to all stations with an equally large update frequency,
the latency between sending the control signal and switching the light should be as short as possible,
the stations should control the lights synchronously and the reliability should still be as high as possible.
The range should be at least 100m and the hardware used should be as cost-effective as possible.

\section*{Motivation/Requirements}
% \begin{itemize}
% 	\item Reliability
% 	\item .. and why 100\% Reliability is not important (except pyrotechnics)
% 	\item Lower latency
% 	\item Synchronisation
% 	\item higher update frequency
% 	\item Range
% \end{itemize}

There will always be a possibility to build an even more reliable system with extremely complex hardware,
however, the focus of this thesis is to achieve results with low-cost hardware that can be compared 
to the existing DMX-512a solution.
The ESP-NOW protocol used here by the manufacturer Espressif is unfortunately proprietary
and had additionally to be investigated due to the partly inaccurate or outdated documentation.
Also, the complete control process finds place over microcontrollers, 
which have to be programmed in a fundamentally different way than ordinary programs.

\section*{Contribution}
% \begin{itemize}
% \item open source available on github [link]
% \item thought-provoking impulse for different approaches
% \item Protocol auf DL Layer/App Layer Ebene
% \item Art-Net baseline/DMX512a
% \item simulativ und experimentel untersucht
% \end{itemize}

A number of protocols are being developed on the DL layer/application layer,
which offer promising improvements.
It is shown that broadcasts are preferable to unicats
and also become significantly more robust through adaptations.
These were investigated simulatively and experimentally and compared with existing solutions.
The test suite used on the test hardware is publicly available on Github.\todo{link in footnote}

\section*{Thesis Outline}

The thesis is divided into three main chapters.
Chapter 3 explains the physical and data link layer, and the 802.11 spzifications family,
existing light protocols are explained in order to better understand the new protocols that will be introduced later,
in addition, the EPS32 hardware and the associated ESP-NOW protocol are briefly shown.

Chapter 4 then specifies the various protocols developed and makes analytical assumptions.
And it is explained how to work with the hardware used.

Chapter 5 then explains the methodology, how the measurement data was collected and the technical implementation.
Wireshark measurements are analysed and used to better understand the protocols.
Afterwards, the protocol data is further evaluated and compared with the analytical approaches and with each other.