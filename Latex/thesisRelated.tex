  
\chapter{Related Work}
% \begin{itemize}
% 	\item Wie der und der in Paper so gezeigt hat 
% 	\item Auch Ding et al haben versucht
% 	\item ...
% 	\item 10 Paper
% 	\item halbe seite
% \end{itemize}


% \begin{itemize}
% 	\item A First Implementation and Evaluation of the IEEE 802.11aa Group Addressed Transmission Service    
% 		\subitem unsosliced Repetition
% 		\subitem blockack
% 	\item Evaluation of Error Control Mechanisms for 802.11b Multicast Transmissions
% 		\subitem packet loss rate
% 		\subitem ARQ, FEC
% 	\item ESP-NOW communication protocol with ESP32
% 		\subitem ESP-NOW details
% 	\item The Working Principles of ESP32 and Analytical Comparision of using Low-Cost Microcontroller Modules in Embedded Systems Design
% 		\subitem why the ESP32 is superior over arduino
% 	\item Adaptive Cross-Layer Protection Strategies for Robust Scqalable Video Transmissions Over 802.11 WLANs
% 	\item Voice Capacity of IEEE 802.11b, 802.11a and 802.11g Wireless LANs
% \end{itemize}

The trend towards wireless solutions in all areas also affects stage technology.
However, wireless solutions often have lower throughput, latency and reliability than their wired counterparts.

Cost-effective solutions work with the 802.11 protocol, which is widely used. 
H. Kareem and D. Dunaev \cite{TheWorkingPrincipalsOfESP32} observed that
\emph{the recently growing demand for the control and automation of a wide variety of devices and gadgets has
led to a rapid expansion in embedded systems market}.
When selecting the platform for the experiments in this thesis, the ESP32 was chosen, 
which was compared analytical along with Arduino embedded systems from the two,
with the result, \emph{to highlight the advantages of the ESP32 in designing embedded systems compared to similar boards}.

Due to the fact that many individual stations are controlled in stage lighting, often with very short signals, there is sometimes an extreme overhead.
\emph{Overhead is} 
according to Y. Xiao \cite{PerformanceEnhancement}
\emph{one of the fundamental problems of MAC inefficiency, and it includes MAC header, frame check sequence, and physical header.}
But also DIFS, SIFS can be be classified as overhead.

The impact of transmission overhead is also researched by Nurul Sarkar \cite{TheImpactOfOverheads}
he pointed out the DCF inefficiency and develpoed a protocol that
\emph{is a simple packet scheduling mechanism that can be used to reduce transmission overheads of DCF and to improve the performance. 
The key idea is to create a temporary buffer unit at the MAC layer for each active connection on the network 
where multiple packets are accumulated nd combined into a single large packet}.
In this thesis where also broadcasts used to address multiple stations in order to reduce overhead compared to multiple unicast transmissions.
Nurul Sarkar \cite{TheImpactOfOverheads} also took a look into 
\emph{its impact on throughput for a single-user wireless ad hoc network},
which where used in this thesis.

Addressing to multiple stations is often realized with multicasts, 
because, unlike broadcasts, acknolegements can be sent. One approach to this is presented by Pablo Salvador et. al. 
\cite{AFirstImplementation}
They experimentally investigate the IEEE 802.11aa GATS and test block acknolegements.
Also unsolicited retries where researched by them: 
\emph{the idea is to improve reliability with a very somple scheme, wich does not requre a 
closed loop between the sender and the receivers(s), and therefore the price to pay is efficiency}.
A method that was also used in this thesis to improve reliability.

Other approaches use PCF to transmit real-time traffic for video or audio transmissions,
for example, Mihaela et al. \cite{AdaptiveCrossLayer} use these 
\emph{since it is the most efective scenario for video transport over 802.11 WLANS.}
But they also say that simple 
\emph{retransmission is not very effective in combating long burst of packet losses},
which are circumvented in this thesis by delayed repetition.


% \section{References}

% What follows is just a very quick refresher on how to use references.
% It is not a guide on scientific writing in general, nor copyright and plagiarism in particular.
% Please refer to an actual guide on technical writing and scientific practices to make sure you understand how, where, and when to cite.

% Simply speaking, proper scientific writing has to deal with two closely related (but not identical) concepts:
% \begin{enumerate}[label=\alph*),ref=(\alph*)]
% \item\label{itm:ref:copy}
% Copyright
% \item
% Plagiarism\label{itm:ref:plag}
% \end{enumerate}
% Do not confuse the need for properly citing your sources as something related to copyright.
% Questions of~\ref{itm:ref:copy} copyright or the corresponding national equivalent deal with who has the right to reproduce a certain text excerpt, an image, or something similar.
% Questions of~\ref{itm:ref:plag} plagiarism deal with who came up with a certain idea or insight, e.g., a certain finding, a certain concept, or a certain way of illustrating a concept.
% By way of analogy, consider a car: after buying a car you have the right to~\ref{itm:ref:copy} do whatever you want with it, but you still cannot claim that you~\ref{itm:ref:plag} invented it.
% Conversely, properly~\ref{itm:ref:plag} crediting who invented your neighbor's car does not give you the right to~\ref{itm:ref:copy} use it.
% Put yet another way, problem~\ref{itm:ref:copy} is a legal one: to be allowed to publish a scientific work you (or, rather, your publisher) needs to have permission to reproduce it -- or suffer legal consequences like heavy fines.
% Problem~\ref{itm:ref:plag} is an academic one: claiming someone else's ideas as one's own is plagiarism; similarly, re-selling old ideas as new ones is self-plagiarism.
% Both incur heavy penalties like exclusion from schools and professional associations or being blacklisted from publishing with scientific outlets for any number of years.

% You will need to address both problems in writing your thesis.
% Problem~\ref{itm:ref:copy} can be addressed in two ways:
% First, by creating original content (that is, text or figures) yourself, which is always preferable as this gives you the freedom to present the content your way.
% Second, by obtaining a license to reproduce content (e.g., by way of buying a license or adhering to the terms of an existing copyleft license).
% Problem~\ref{itm:ref:plag} can be addressed in two ways:
% First, presenting original ideas and insights (as you will do when presenting own results).
% Second, by clearly pointing out the (primary) source of an idea.
% The latter is the topic of this section.

% In brief, use references whenever you cite from related work (either directly or indirectly), or when you build on related work (this includes their way of illustrating a particular concepts, in text form as well as in the overall design of a figure).
% Also use references to point a reader to related work.
% Clearly distinguish between these uses.
% Make it very clear which part of a statement a reference belongs to.
% Compare the following three, vastly different uses (where the cited idea appears in \textbf{boldface}):

% \begin{itemize}
% \item ``\textbf{Foo and bar are of equal value. Thus, any can be used.}''~\cite{akyildiz2002survey,arampatzis2005survey}
% \item According to~\cite{akyildiz2002survey} and~\cite{arampatzis2005survey}, \textbf{foo and bar are of equal value, and any can be used}.
% \end{itemize}
% versus
% \begin{itemize}
% \item ``\textbf{Foo and bar are of equal value}''~\cite{akyildiz2002survey,arampatzis2005survey}. Thus, any can be used.
% \item According to~\cite{akyildiz2002survey} and~\cite{arampatzis2005survey}, \textbf{foo and bar are of equal value}. From this it follows that any can be used.
% \end{itemize}
% versus
% \begin{itemize}
% \item Foo and \textbf{bar}~\cite{akyildiz2002survey,arampatzis2005survey} are of equal value. Thus, any can be used.
% \item Foo and \textbf{bar} (detailed in~\cite{akyildiz2002survey} and~\cite{arampatzis2005survey}) are of equal value. Thus, any can be used.
% \end{itemize}
% versus
% \begin{itemize}
% \item \textbf{Foo}~\cite{akyildiz2002survey} and \textbf{bar}~\cite{arampatzis2005survey} are of equal value. Thus, any can be used.
% \item \textbf{Foo} (detailed in~\cite{akyildiz2002survey}) and \textbf{bar} (detailed in~\cite{arampatzis2005survey}) are of equal value. Thus, any can be used.
% \end{itemize}

% Never typeset a reference after the final full stop of a paragraph (or sentence) and expect your reader to figure out which part of the paragraph is an indirect citation and which part is original (i.e., your own) work.
% When paraphrasing longer passages of text, use an indirect citation.
% Make sure to clearly point out when you are finished paraphrasing, like so:
% \emph{According to \textcite{akyildiz2002survey}, Foo and Bar can be characterized as follows. They are big. They are bright. The authors further argue that one can be substituted for the other. In the following I will go on to prove that this is not true.}

% When citing more than a few pages worth of text, point the reader to the specific part you are referring to in your citation, like so:
% \emph{In recent years, an increasing number of cyclists are switching from air filled tires to cement filled ones~\cite[Table IV]{dietrich2009lifetime}}.

% If a figure or a table is closely based on another one, make sure to cite its source, preferably in its caption, like so:
% \emph{Figure 1 -- the relation of ravens and writing desks (based on~\cite[Figure~42]{dietrich2009lifetime})}.
% Be aware that, while there is a well-established convention on how to illustrate a verbatim quote of text (by using quotation marks), there is no well-established convention for indicating that an image was copied verbatim.
% Thus, when citing a figure or table, you must explicitly state whether it was copied verbatim, ed, or whether it served as inspiration for your own.

% Do not cite URLs. Content found there is not peer reviewed and it is likely to change during the lifetime of your work.
% For pointing a reader to interesting websites, use footnotes -- but trust your reader to know how to use a web search engine.

% Your text reads nicer if you do not use citations as a substitute for nouns (like this section did).
% Instead of \emph{The benefits of cement filled tires has been shown by \cite{akyildiz2002survey}}, consider writing \emph{\textcite{akyildiz2002survey} have shown the benefits of cement filled tires}.
% The \texttt{textcite} command makes this straightforward.

% Make sure to read your bibliography section (that is, the typeset list of references) after you are done adding all citations to your text.
% Does it contain all information needed to uniquely identify to references you used?
% Do not trust BibTeX files you find on the web:
% Digital libraries frequently have their contents wrong, are missing information, or are using different field names than your bibliography style expects (leading to missing information in the typeset bibliography).
% To give a few examples:
% Check the authors' list (making sure all authors are listed in the same order and in the same way they are listed in the publication).
% Check the conference location (it's most likely not ``New York, New York'').
% Check the publisher name (many digital libraries use a field that is not typeset by your bibliography style; have a look at the demo bibliography in this template for how to deal with that).
% Check the page numbers (many digital libraries put ``1--5'' here despite the paper starting at a later page -- or despite it not having any page numbers to begin with).
% Check the conference name, put its parts in a logical order, and lose the ``in proceedings of'' (it's not ``Mobicom, in proceedings of, 1999 series MobiCom99'' but ``5th ACM International Conference on Mobile Computing and Networking (MobiCom 1999)''.\todo{triple-check all references}