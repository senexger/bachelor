\documentclass[12 pt, a4paper, ngerman]{article}

\usepackage[utf8]{inputenc}
\usepackage{subfigure}
\usepackage[paper=a4paper,left=25mm,right=25mm,top=25mm,bottom=25mm]{geometry}
\usepackage[german]{babel}
\usepackage[usenames,dvipsnames]{color}
\usepackage{graphicx}
%\usepackage{verbatim}
\usepackage{fancyhdr}
%\usepackage{textcomp} 
%\usepackage{times}
\usepackage{float}
\usepackage{amsmath}
% \usepackage{amssymb}
\usepackage{exscale}
\usepackage{subfig} 					
\usepackage{booktabs}
\usepackage{ulem}
\usepackage{color} 
\usepackage{listings}

% needed for plotting experiment
\usepackage{pgfplots}       % <-- required in preamble
\usepackage{SIunits}        % <-- required in preamble
\pgfplotsset{compat=newest} % <-- optional in preamble

\usepackage{tikz}
\usepackage{pgfplots}
\pgfplotsset{width=15cm,compat=1.9}

\definecolor{mygreen}{rgb}{0,0.6,0}
\definecolor{mygray}{rgb}{0.5,0.5,0.5}
\definecolor{mymauve}{rgb}{0.58,0,0.82}

\lstset{ %
  backgroundcolor=\color{white},   % choose the background color; you must add \usepackage{color} or \usepackage{xcolor}
  basicstyle=\footnotesize,        % the size of the fonts that are used for the code
  breakatwhitespace=false,         % sets if automatic breaks should only happen at whitespace
  breaklines=true,                 % sets automatic line breaking
  captionpos=b,                    % sets the caption-position to bottom
  commentstyle=\color{mygreen},    % comment style
  deletekeywords={...},            % if you want to delete keywordFs from the given language
  escapeinside={\%*}{*)},          % if you want to add LaTeX within your code
  extendedchars=true,              % lets you use non-ASCII characters; for 8-bits encodings only, does not work with UTF-8
  frame=single,	                   % adds a frame around the code
  keepspaces=true,                 % keeps spaces in text, useful for keeping indentation of code (possibly needs columns=flexible)
  keywordstyle=\color{blue},       % keyword style
  language=Octave,                 % the language of the code
  otherkeywords={*,...},           % if you want to add more keywords to the set
  numbers=left,                    % where to put the line-numbers; possible values are (none, left, right)
  numbersep=5pt,                   % how far the line-numbers are from the code
  numberstyle=\tiny\color{mygray}, % the style that is used for the line-numbers
  rulecolor=\color{black},         % if not set, the frame-color may be changed on line-breaks within not-black text (e.g. comments (green here))
  showspaces=false,                % show spaces everywhere adding particular underscores; it overrides 'showstringspaces'
  showstringspaces=false,          % underline spaces within strings only
  showtabs=false,                  % show tabs within strings adding particular underscores
  stepnumber=2,                    % the step between two line-numbers. If it's 1, each line will be numbered
  stringstyle=\color{mymauve},     % string literal style
  tabsize=2,	                   % sets default tabsize to 2 spaces
  title=\lstname                   % show the filename of files included with \lstinputlisting; also try caption instead of title
}

\usepackage{relsize}  
%\usepackage[pdftex,colorlinks,citecolor = {blue},linkcolor = {blue},urlcolor = {red},a4paper,bookmarks=true,bookmarksopen = true,bookmarksnumbered = true,]{hyperref}
\usepackage{tabulary}
%\marginsize{3cm}{2cm}{2cm}{2cm}

\author{Maximilian Gotthardt (350464)}
\date{\today\\*[30pt]}

%-----------------------------------------------------------------------------
\begin{document}
%-----------------------------------------------------------------------------
\pagenumbering{arabic}
%-----------------------------------------------------------------------------

% \maketitle  % erstellt Dokument-Kopf mit Titel und Authoren wie oben definiert
\thispagestyle{empty} %keine Nummerierung auf der Titelseite

\newpage

% \tableofcontents % Inhaltsverzeichnis
% \newpage
\pagenumbering{arabic}

%---------------------------------------------------------------------
\section{Unicast vs Broadcast}

Channelzahl pro Node = 20 (entspricht 20 Byte Payload). \\

\begin{tikzpicture}
    \begin{axis}[
        title={Unicast vs. Multicast},
        xlabel={Channel},
        ylabel={Latency in µs},
        xmin=0, xmax=600,
        ymin=0, ymax=10000,
        xtick={0,200,400,600},
        ytick={0,1000,2000,3000,4000,5000,6000,7000,8000,9000,10000},
        legend pos=north west,
        ymajorgrids=true,
        grid style=dashed,
    ]

    \addplot[
        color=blue,
        mark=square,
        ]
        coordinates {
        ( 20  , 1189      )
        ( 100 , 6258      )
        ( 160 , 9845      )
        ( 200 , 14499     )
        ( 400 , 34113     )
        };
        \legend{Broadcast}

    \addplot[
        color=red,
        mark=square,
        ]
        coordinates {
        ( 20  , 847    )
        ( 40  , 3981       )
        ( 60  , 3981       )
        ( 80  , 3981       )
        ( 100 , 3981       )
        ( 120 , 3981       )
        ( 140 , 3981       )
        ( 160 , 3981       )
        ( 180 , 3981       )
        ( 200 , 5311       )
        ( 399 , 5311       )
        ( 400 , 5311       )
        ( 599 , 5311       )
        ( 600 , 3551       )
        ( 650 , 3551       )
        ( 700 , 3551       )
        ( 750 , 3551       )
        ( 800 , 3551       )
        };
        \legend{Unicast}
    \end{axis}
\end{tikzpicture}


\newpage
\section{Unicast Abweichung}

Master-Slave-Unicast airtime one Node \\

\begin{tikzpicture}
\begin{axis}[
	xlabel=Repitition,
	ylabel=Airtime (us),
	title={Airtime over time},
	grid=both,
	minor grid style={gray!25},
	major grid style={gray!25},
	width=0.75\linewidth,
	no marks]
\addplot[
    line width=1pt,
    solid,
    color=blue,
    mark=square,
    ]
	table[x=R,y=u20c,col sep=comma]{data/TimePerSendOver100.csv};
\addlegendentry{Master-Slave Unicast};
\addplot[
    line width=1pt,
    solid,
    color=red,
    mark=square,
    ]
	table[x=R,y=b20c,col sep=comma]{data/TimePerSendOver100.csv};
\addlegendentry{Master-Slave Broadcast};
\end{axis}
\end{tikzpicture}

\newpage
\section{Broadcast Abweichung}

Master-Slave-Broadcast airtime one Node \\

\begin{tikzpicture}
\begin{axis}[
	xlabel=Repitition,
	ylabel=Airtime (us),
	title={Airtime over time},
	grid=both,
	minor grid style={gray!25},
	major grid style={gray!25},
	width=0.75\linewidth,
	no marks]
\addplot[
    line width=1pt,
    solid,
    color=red,
    mark=square,
    ]
	table[x=R,y=b20c,col sep=comma]{data/TimePerSendOver100.csv};
\addlegendentry{Master-Slave Broadcast};
\end{axis}
\end{tikzpicture}

\newpage
\section{Sending to multiple nodes}
Master-Slave-Broadcast airtime one Node \\

\begin{tikzpicture}
\begin{axis}[
	xlabel=Repitition,
	ylabel=Airtime (us),
	title={Airtime over time},
	grid=both,
	minor grid style={gray!25},
	major grid style={gray!25},
	width=0.75\linewidth,
	no marks]
\addplot[
    line width=1pt,
    solid,
    color=red,
    mark=square,
    ]
	table[x=R,y=u_c60n3,col sep=comma]{data/TimePerSendOver100.csv};
\addlegendentry{Master-Slave Broadcast};
\end{axis}
\end{tikzpicture}

\newpage
\section{Unicast increasing Nodes}
Master-Slave-Broadcast airtime one Node \\

\begin{tikzpicture}
\begin{axis}[
	xlabel=Repitition,
	ylabel=Latency (us),
	title={Increasing Nodes over time},
	grid=both,
	minor grid style={gray!25},
	major grid style={gray!25},
	width=0.75\linewidth,
	no marks]
\addplot[
    line width=1pt,
    solid,
    color=red,
    mark=square,
    ]
	table[x=R,y=u_c60n3,col sep=comma]{data/TimePerSendOver100.csv};
\addlegendentry{3 Nodes};
\addplot[
    line width=1pt,
    solid,
    color=blue,
    mark=square,
    ]
	table[x=R,y=u_c80n4,col sep=comma]{data/TimePerSendOver100.csv};
\addlegendentry{4 Nodes};
\addplot[
    line width=1pt,
    solid,
    color=green,
    mark=square,
    ]
	table[x=R,y=u_c200n10,col sep=comma]{data/TimePerSendOver100.csv};
\addlegendentry{10 Nodes};
\addplot[
    line width=1pt,
    solid,
    color=yellow,
    mark=square,
    ]
	table[x=R,y=u_c400n20,col sep=comma]{data/TimePerSendOver100.csv};
\addlegendentry{20 Nodes};
\end{axis}
\end{tikzpicture}




\newpage
\section{Broadcast increasing Nodes}
Master-Slave-Broadcast airtime one Node \\

\begin{tikzpicture}
\begin{axis}[
	xlabel=Repitition,
	ylabel=Latency (us),
	title={Increasing Nodes over time},
	grid=both,
	minor grid style={gray!25},
	major grid style={gray!25},
	width=0.75\linewidth,
	no marks]
\addplot[
    line width=1pt,
    solid,
    color=red,
    mark=square,
    ]
	table[x=R,y=b_c20n1,col sep=comma]{data/TimePerSendOver100.csv};
\addlegendentry{1 Nodes};
\addplot[
    line width=1pt,
    solid,
    color=blue,
    mark=square,
    ]
	table[x=R,y=b_c200n10,col sep=comma]{data/TimePerSendOver100.csv};
\addlegendentry{10 Nodes};
\addplot[
    line width=1pt,
    solid,
    color=green,
    mark=square,
    ]
	table[x=R,y=b_c400n20,col sep=comma]{data/TimePerSendOver100.csv};
\addlegendentry{20 Nodes};
\addplot[
    line width=1pt,
    solid,
    color=black,
    mark=square,
    ]
	table[x=R,y=b_c600n30,col sep=comma]{data/TimePerSendOver100.csv};
\addlegendentry{30 Nodes};
\end{axis}
\end{tikzpicture}

\end{document}
